\documentclass{article}
\usepackage[utf8]{inputenc}
\usepackage{amsmath}
\title{Algebra Worksheet}
\author{Simeon Chester}
\date{2021-10-01}
\begin{document}
\maketitle
\begin{description}
    \item[A.]  Simplify
        \begin{enumerate}
            \item $ (6x^2 - 10xy + 2) + (2z - xy + 4)$
                  \begin{equation}
                      \begin{split}
                          & = 6x^2 - 10xy + 2 + 2z - xy + 4 \\
                          & = 6x^2 - 10xy - xy + 2z + 2 + 4 \\
                          & = 6x^2 -11xy + 2z + 6 \\
                      \end{split}
                  \end{equation}
            \item $ 4(2z-w) - 3(w-2z) $
                  \begin{equation}
                      \begin{split}
                          & = 8z - 4w -3w -6z \\
                          & = -4w - 3w + 8z - 6z \\
                          & = -7w + 2z \\
                      \end{split}
                  \end{equation}
            \item $ (8t^2 - 6x^2) + (4s^2 - 2t^2 + 6) $
                  \begin{equation}
                      \begin{split}
                          & = 8t^2 - 6x^2 + 4s^2 - 2t^2 + 6 \\
                          & = 4s^2 + 8t^2 - 2t^2 - 6x^2 + 6 \\
                          & = 4s^2 - 6t^2 - 6x + 6 \\
                      \end{split}
                  \end{equation}
            \item $ (3a - 7b - 9) - (5a + 9b + 21) $
                  \begin{equation}
                      \begin{split}
                          & = 3a - 7b - 9  - 5a - 9b - 21 \\
                          & = 3a - 5a -7b - 9b - 9 - 21 \\
                          & = -2a - 16b - 30 \\
                      \end{split}
                  \end{equation}
            \item $ 2 - [3 + 4(p-3)] $
                  \begin{equation}
                      \begin{split}
                          & = 2 - [3 + 4p - 12] \\
                          & = 2 - 3 - 4p + 12 \\
                          & = -1 - 4p + 12 \\
                          & = -4p -1 + 12 \\
                          & = -4p + 11 \\
                      \end{split}
                  \end{equation}
            \item $(2p - 1) (2p + 1)$
                  \begin{equation}
                      \begin{split}
                          & = 2p(2p+1) -1(2p+1) \\
                          & = 4p^2 + 2p - 2p - 1 \\
                          & = 4p^2 - 1 \\
                      \end{split}
                  \end{equation}
            \item $  (x^2 + x + 1)^2 $
                  \begin{equation}
                      \begin{split}
                          & = (x^2 + x + 1)(x^2 + x + 1) \\
                          & = x^2(x^2 + x + 1)+x(x^2 + x + 1)+1(x^2 + x + 1) \\
                          & = x^4 + x^3 + x^2 + x^3 + x^2 + x + x^2 + x + 1 \\
                          & = x^4 + x^3 + x^3 + x^2 + x^2 + x^2 + x + x + 1 \\
                          & = x^4 + 2x^3 + 3x^2 + 2x + 1 \\
                      \end{split}
                  \end{equation}
            \item $ \frac{x^3} {(x+2)} $
                  \begin{equation}
                      \begin{split}
                          &  = \frac{(x^3) \times (x-2)}{(x+2) \times (x-2)} \\
                          &  = \frac{x^4 - 2x^3}{x(x-2)+2(x-2)} \\
                          &  = \frac{x^4 - 2x^3}{x^2-2x+2x-4} \\
                          &  = \frac{x^4 - 2x^3}{x^2-4} \\
                      \end{split}
                  \end{equation}
            \item $ \frac{2x^3-7x+4}{x}$
                  \begin{equation}
                      \begin{split}
                          & \\
                      \end{split}
                  \end{equation}
            \item $ \frac{6x^5 + 4x^3 - 1}{2x^2}$
                  \begin{equation}
                      \begin{split}
                          & \\
                      \end{split}
                  \end{equation}
        \end{enumerate}
    \item[B. ] Factor Completely
        \begin{enumerate}
            \item $ 2ax - 2b $
                  \begin{equation}
                      \begin{split}
                          & = 2(ax-b)\\
                      \end{split}
                  \end{equation}
            \item $ z^2 - 49 $
                  \begin{equation}
                      \begin{split}
                          & = (z)^2 - (7)^2\\
                          Since (a+b)(a-b) = a^2 - b^2 \\
                          & = (z+7)(z-7)\\
                      \end{split}
                  \end{equation}
            \item $ 16x^2 - 9 $
                  \begin{equation}
                      \begin{split}
                          & = (4x)^2 - (3)^2 \\
                          Since (a+b)(a-b) = a^2 - b^2 \\
                          & = (4x - 3)(4x+3)\\
                      \end{split}
                  \end{equation}
            \item $ 3x^2 - 3 $
                  \begin{equation}
                      \begin{split}
                          & = 3(x^2-1)\\
                          & = 3((x)^2-(1)^2)\\
                          Since (a+b)(a-b) = a^2 - b^2 \\
                          & = 3((x-1)(x+1))\\
                      \end{split}
                  \end{equation}
            \item $ x^2 + 2x - 24 $
                  \begin{equation}
                      \begin{split}
                          & = x^2 + 6x - 4x - 24\\
                          & = x(x+6)-4(x + 6)\\
                          & = (x-4)(x+6)\\
                      \end{split}
                  \end{equation}
            \item $ 4x^2 - x - 3 $
                  \begin{equation}
                      \begin{split}
                          & = 4x^2 -4x + 3x - 3\\
                          & = 4x(x-1)+3(x-1)\\
                          & = (4x+3)(x-1)\\
                      \end{split}
                  \end{equation}
            \item $ (4x + 2)^2 $
                  \begin{equation}
                      \begin{split}
                          & = (4x+2)(4x+2) \\
                          & = 4x(4x+2) +2(4x+2)\\
                          & = 16x^2 +8x +8x +4\\
                          & = 16x^2 + 16x + 4 \\
                      \end{split}
                  \end{equation}
            \item $ 2x^2(2x-4x^2)^2 $
                  \begin{equation}
                      \begin{split}
                          & = 2x^2(4x^2 - 16x^4)\\
                          & = 8x^4 - 32x^6\\
                      \end{split}
                  \end{equation}
        \end{enumerate}
    \item[C. ] Simplify
        \begin{enumerate}
            \item $\frac{a^2 - 9}{a^3-3a}$
                  \begin{equation}
                      \begin{split}
                          & = \frac{(a)^2 - (3)^2}{a(a^2 - 3)}\\
                          & = \frac{(a+3)(a-3)}{a(a^2-3)}\\
                      \end{split}
                  \end{equation}
            \item $\frac{x^2 - 3x - 10}{x^2-4}$
                  \begin{equation}
                      \begin{split}
                          & = \frac{x^2 -5x + 2x - 10}{(x)^2-(2)^2}\\
                          & = \frac{x(x-5)+2(x-5)}{(x+2)(x-2)} \\
                          & = \frac{(x+2)(x-5)}{(x+2)(x-2)}\\
                          & = \frac{\frac{(x+2)(x-5)}{x+2}} {\frac{(x+2)(x-2)}{x+2}}\\
                          & = \frac{x-5}{x-2}\\
                      \end{split}
                  \end{equation}
            \item $\frac{6x^2 + x - 2}{2x^2 + 3x - 2}$
                  \begin{equation}
                      \begin{split}
                          & = \frac{6x^2+4x-3x-2}{2x^2+4x-x-2}\\
                          & = \frac{2x(3x+2)-1(3x+2)}{2x(x+2)-1(x+2)} \\
                          & = \frac{(2x-1)(3x+2)}{(2x-1)(x+2)}\\
                          & = \frac{\frac{(2x-1)(3x+2)}{2x-1}}{\frac{(2x-1)(x+2)}{2x-1}}\\
                          & = \frac{3x+2}{x+2}\\
                      \end{split}
                  \end{equation}
            \item $(\frac{y^2}{y-3})(\frac{-1}{y+2})$
                  \begin{equation}
                      \begin{split}
                          & = \frac{(y^2)(-1)}{(y-3)(y+2)})\\
                          & = \frac{y^2}{y(y+2)-3(y+2)}\\
                          & = \frac{y^2}{y^2+2y-3y-6}\\
                          & = \frac{y^2}{y^2-y-6}\\
                          & = \frac{\frac{y^2}{y^2}}{\frac{y^2-y-6}{y^2}}\\
                          & = \frac{1}{-y-6}
                      \end{split}
                  \end{equation}
            \item $(\frac{ax-b}{x-c})(\frac{c-x}{ax+b})$
                  \begin{equation}
                      \begin{split}
                          & = \frac{(ax-b)(c-x)}{(x-c)(ax+b)}\\
                          & = \frac{ax(c-x)-b(c-x)}{x(ax+b)-c(ax+b)}\\
                          & = \frac{acx - ax^2 -bc +bx}{ax^2 +bx -acx - bc}\\
                          & = \frac{-ax^2 + acx -bc +bx}{ax^2 - acx +bx -bc}\\
                          & = \\
                      \end{split}
                  \end{equation}
            \item $\frac{4}{a+4}+a$
                  \begin{equation}
                      \begin{split}
                          & = \frac{4}{a+4}+ \frac{a}{1}\\
                          & = \frac{4 + a(a+4)}{a+4}\\
                          & = \frac{4+a^2+4a}{a+4}\\
                          & = \frac{a^2+4a+4}{a+4}\\
                      \end{split}
                  \end{equation}
            \item $\frac{x^2}{x+3}+\frac{5x+6}{x+3}$
                  \begin{equation}
                      \begin{split}
                          & = \frac{x^2+5x+6}{x+3}\\
                          & = \frac{x^2 + 3x + 2x + 6}{x+3}\\
                          & = \frac{x(x+3)+2(x+3)}{x+3}\\
                          & = \frac{\frac{x(x+3)}{x+3}+\frac{2(x+3)}{x+3}}{\frac{x+3}{x+3}}\\
                          & = \frac{x+2}{1}\\
                          & = x+2 \\
                      \end{split}
                  \end{equation}
            \item $\frac{\frac{x^2+6x+9}{x}}{x+3}$
                  \begin{equation}
                      \begin{split}
                          & = \frac{x^2+6x+9}{x} \div \frac{x+3}{1}\\
                          & = \frac{x^2+6x+9}{x} \times \frac{1}{x+3}\\
                          & = \frac{x^2 + 6x + 9}{x^2 + 3x}\\
                          & = \frac{x^2 +3x + 3x +9}{x^2+3x}\\
                          & = \frac{x(x+3)+3(x+3)}{x(x+3)}\\
                          & = \frac{(x+3)(x+3)}{x(x+3)} \\
                          & = \frac{\frac{(x+3)(x+3)}{x+3}}{\frac{x(x+3)}{x+3}} \\
                          & = \frac{x+3}{x}\\
                      \end{split}
                  \end{equation}
            \item $\frac{\frac{4x}{3}}{2x}$
                  \begin{equation}
                      \begin{split}
                          & = \frac{4x}{3} \div \frac{2x}{1}\\
                          & = \frac{4x}{3} \times \frac{1}{2x}\\
                          & = \frac{4x}{6x}\\
                          & = \frac{\frac{4x}{2}}{\frac{6x}{2}} \\
                          & = \frac{2x}{3x}\\
                          & = \frac{2}{3}\\
                      \end{split}
                  \end{equation}
            \item $\frac{7 + \frac{1}{x}}{5}$
                  \begin{equation}
                      \begin{split}
                          & = (7+\frac{1}{x}) \div \frac{5}{1}\\
                          & = (\frac{7x}{x}+\frac{1}{x}) \times \frac{1}{5}\\
                          & = \frac{7x+1}{x} \times \frac{1}{5}\\
                          & = \frac{7x+1}{5x}\\
                      \end{split}
                  \end{equation}
        \end{enumerate}
    \item[D. ] Solve for $x$
        \begin{enumerate}
            \item $7x + 7 = 2(x+1)$
                  \begin{equation}
                      \begin{split}
                          7x + 7 & = 2x + 1\\
                          7x - 2x & = 1 - 7\\
                          5x & = -6 \\
                          x  & = \frac{-6}{5}\\
                      \end{split}
                  \end{equation}
            \item $5(p-7)-2(3p-4) = 3p$
                  \begin{equation}
                      \begin{split}
                          5p - 35 - 6p + 8 & = 3p \\
                          5p - 6p - 3p & = 35 - 8 \\
                          -4p & = 27 \\
                          p & = -\frac{27}{4} \\
                      \end{split}
                  \end{equation}
            \item $\frac{5}{x} = 25$
                  \begin{equation}
                      \begin{split}
                          5 & = 25x \\
                          25x & = 5 \\
                          x & = \frac{5}{25}\\
                          x & = \frac{1}{5}\\
                      \end{split}
                  \end{equation}
            \item $\frac{5}{3-x} = 0$
                  \begin{equation}
                      \begin{split}
                      \end{split}
                  \end{equation}
            \item $\frac{x+3}{x} = \frac{2}{5}$
                  \begin{equation}
                      \begin{split}
                          5(x+3) & = 2(x)\\
                          5x+15 & = 2x \\
                          5x-2x & = -15 \\
                          3x & = -15 \\
                          x & = -\frac{15}{3}\\
                          x & = -5 \\
                      \end{split}
                  \end{equation}
            \item $\frac{3}{5-x} = \frac{7}{2}$
                  \begin{equation}
                      \begin{split}
                          2(3) & = 7(5-x)\\
                          6 & = 35-7x\\
                          7x & = 35 - 6 \\
                          7x & = 29 \\
                          x & = \frac{29}{7}\\
                      \end{split}
                  \end{equation}
            \item $\frac{2x-3}{4x-5} = 6$
                  \begin{equation}
                      \begin{split}
                          6(4x-5) & = 1(2x-3)\\
                          24x-30 & = 2x-3\\
                          24x-2x & = 30-3\\
                          22x & = 27\\
                          x &=\frac{27}{22}\\
                      \end{split}
                  \end{equation}
            \item $\frac{1}{x} + \frac{1}{7} = \frac{3}{7}$
                  \begin{equation}
                      \begin{split}
                          \frac{1}{x} & = \frac{3}{7} - \frac{1}{7}\\
                          \frac{1}{x} & = \frac{2}{7} \\
                          2(x) & = 1(7) \\
                          2x & = 7 \\
                          x & = \frac{7}{2} \\
                      \end{split}
                  \end{equation}
            \item $\frac{2}{x-1} = \frac{3}{x-2}$
                  \begin{equation}
                      \begin{split}
                          2(x-2) & = 3(x-1)\\
                          2x-4 & = 3x - 3 \\
                          2x - 3x & = -3 + 4 \\
                          -x & = 1 \\
                          -x \times -1 & = 1 \times -1 \\
                          x & = -1\\
                      \end{split}
                  \end{equation}
            \item $\sqrt{x+5} = 4$
                  \begin{equation}
                      \begin{split}
                          \sqrt{x+5}^2& = 4 ^ 2\\
                          x+5 & = 16 \\
                          x & = 16  - 5 \\
                          x & = 11 \\
                      \end{split}
                  \end{equation}
            \item $(x+6)^\frac{1}{2} = 7$
                  \begin{equation}
                      \begin{split}
                          ((x+6)^\frac{1}{2})^2 & = 7^2 \\
                          x + 6 & = 49 \\
                          x & = 49 - 6 \\
                          x & = 43 \\
                      \end{split}
                  \end{equation}
        \end{enumerate}

    \item[E. ] Express the indicated symbol in terms of the remaining symbols. \\
        Example: If $s = \frac{u}{au+v}$, express $u$ in terms of the others, i.e. find for $u$. \\
        by cross multiplying:
        \begin{equation}
            \begin{split}
                & s(au + v) = u\\
                & sau + sv = u\\
                & sau - u = -sv\\
                & u(sa - 1) = -sv\\
                & u = \frac{-sv}{sa-1}\\
            \end{split}
        \end{equation}
        To do
        \begin{enumerate}
            \item $p = -3q+6$, find $q$
                  \begin{equation}
                      \begin{split}
                          p & = -3q + 6 \\
                          3q & = -p + 6 \\
                          q & = \frac{-p+6}{3} \\
                      \end{split}
                  \end{equation}
            \item $s = P(1+rt)$, find $r$
                  \begin{equation}
                      \begin{split}
                          s & = P(1+rt) \\
                          s & = P +Prt \\
                          -Prt & = P - s \\
                          r & = \frac{P - s}{-Pt}
                      \end{split}
                  \end{equation}
            \item $\frac{2mI}{B(n+1)}$, find $I$
                  \begin{equation}
                      \begin{split}
                      \end{split}
                  \end{equation}
            \item $\frac{d}{1+dt}$, find $t$
        \end{enumerate}
    \item[F. ] Solve by factoring.
        \begin{enumerate}
            \item $t^2 - 8t + 15 = 0$
                  \begin{equation}
                      \begin{split}
                          t^2 - 8t + 15 & = 0 \\
                          t^2 - 5t -3t + 15 & = 0 \\
                          t(t - 5)-3(t-5) & = 0 \\
                          (t-3)(t-5) & = 0 \\
                      \end{split}
                  \end{equation}
            \item $ -x^2 + 3x + 10 = 0 $
                  \begin{equation}
                      \begin{split}
                          -x^2 + 3x + 10 & = 0 \\
                          -x^2 + 5x - 2x + 10 & = 0 \\
                          -x(x - 5) -2 (x - 5) & = 0 \\
                          (-x - 2) (x - 5) = 0
                      \end{split}
                  \end{equation}
            \item $2b^2 + 9b = 5$
                  \begin{equation}
                      \begin{split}
                          2b^2+ 9b - 5 & = 0 \\
                          2b^2 + 10b - b - 5 & = 0 \\
                          2b(b+5) - 1(b+5) & = 0 \\
                          (2b -1 ) (b+5) & = 0 \\
                      \end{split}
                  \end{equation}
        \end{enumerate}

    \item[G. ] Solve by using the quadratic formula.
        \begin{enumerate}
            \item $x^2 + 2x - 24 = 0$
                  \begin{equation}
                      \begin{split}
                          &a = 1, \ \ b = 5, \ \ c=6 \\
                          x = & \frac{-b\pm \sqrt{b^2 - 4ac}}{2a}\\
                          x = & \frac{-5\pm \sqrt{5^2 - 4(1)(6)}}{2(1)}\\
                          x = & \frac{-5\pm \sqrt{25 - 24}}{2}\\
                          x = & \frac{-5\pm \sqrt{1}}{2}\\
                      \end{split}
                  \end{equation}
                  Therefore $x$ = -2 OR $x$ = -3
            \item $q^2 - 5q = 0$
                  \begin{equation}
                      \begin{split}
                          & q^2 - 5q - 0 = 0\\
                          &a = 1, \ \ b = 5, \ \ c = 0 \\
                          x = & \frac{-b\pm \sqrt{b^2 - 4ac}}{2a}\\
                          x = & \frac{-5\pm \sqrt{5^2 - 4(1)(0)}}{2(1)}\\
                          x = & \frac{-5\pm \sqrt{25 - 0}}{2}\\
                          x = & \frac{-5\pm \sqrt{25}}{2}\\
                      \end{split}
                  \end{equation}
                  Therefore $x$ = 0 OR $x$ = -5
            \item $-2x^2 -6x + 5 = 0$
                  \begin{equation}
                      \begin{split}
                          &a = 2, \ \ b = 6, \ \ c = 5 \\
                          x = & \frac{-b\pm \sqrt{b^2 - 4ac}}{2a}\\
                          x = & \frac{-6\pm \sqrt{6^2 - 4(2)(5)}}{2(2)}\\
                          x = & \frac{-6\pm \sqrt{36 - 40}}{4}\\
                          x = & \frac{-6\pm \sqrt{-4}}{4}\\
                      \end{split}
                  \end{equation}
            \item $2 - 2x + x^2 = 0$
                  \begin{equation}
                      \begin{split}
                          &x^2 - 2x + 2 = 0\\
                          &a = 1, \ \ b = 2, \ \ c = 2 \\
                          x = & \frac{-b\pm \sqrt{b^2 - 4ac}}{2a}\\
                          x = & \frac{-2\pm \sqrt{2^2 - 4(1)(2)}}{2(1)}\\
                          x = & \frac{-2\pm \sqrt{4 - 8}}{2}\\
                          x = & \frac{-2\pm \sqrt{-4}}{2}\\
                      \end{split}
                  \end{equation}
        \end{enumerate}
    \item[H. ] Solve the systems algebraically.
        \begin{enumerate}
            \item
                  \begin{gather*}
                      x+4y = 3 \ \ (eq. \ 1)\\
                      3x-2y=-5 \ \ (eq. \ 2)
                  \end{gather*}
                  Finding $x$ in equation 1:
                  \begin{equation}
                      \begin{split}
                          x + 4y & = 3 \\
                          x      & = 3 - 4y \\
                      \end{split}
                  \end{equation}
                  Substituting $x = 3 - 4y$ in equation 2:
                  \begin{equation}
                      \begin{split}
                          3x - 2y & = -5 \\
                          3(3-4y) - 2y & = -5 \\
                          9 - 12y -2y & = -5 \\
                          9 -14y & = -5 \\
                          -14y & = -5 -9 \\
                          -14y & = -14 \\
                          y & = \frac{-14}{-14} \\
                          y & = 1 \\
                      \end{split}
                  \end{equation}
                  Substituting $y=1$ in equation 1:
                  \begin{equation}
                      \begin{split}
                          x + 4y & = 3 \\
                          x + 4(1) & = 3\\
                          x + 4 & = 3\\
                          x & = 3-4\\
                          x & = -1\\
                      \end{split}
                  \end{equation}
                  Therefore $x=-1$ and $y=1$.
            \item
                  \begin{gather*}
                      x-2y = -7 \ \ (eq. \ 1) \\
                      5x+3y=-9 \ \ (eq. \ 2)
                  \end{gather*}
                  Finding $x$ in equation 1:
                  \begin{equation}
                      \begin{split}
                          x - 2y & = -7 \\
                          x & = 2y - 7 \\
                      \end{split}
                  \end{equation}
                  Substituting $x = 2y - 7$ in equation 2:
                  \begin{equation}
                      \begin{split}
                          5x + 3y      & = -9 \\
                          5(2y-7) + 3y & = -9 \\
                          10y - 35 + 3y & = -9 \\
                          10y + 3y - 35 & = -9 \\
                          13y           & = -9 + 35 \\
                          13y           & = 26 \\
                          y             & = \frac{26}{13} \\
                          y             & = 2 \\
                      \end{split}
                  \end{equation}
                  Substituting $y = 2$ in equation 1:
                  \begin{equation}
                      \begin{split}
                          x - 2y & = -7 \\
                          x - 2(2) & = -7 \\
                          x - 4 & = -7 \\
                          x &= -7 + 4 \\
                          x &= -3
                      \end{split}
                  \end{equation}
                  Therefore $x=-3$ and $y=2$.
            \item
                  \begin{gather*}
                      4x - 3y -2 = 3x - 7y \ \ (eq. \ 1)\\
                      x + 5y - 2 = y + 4 \ \ (eq. \ 2)
                  \end{gather*}
                  Finding $x$ in equation 1:
                  \begin{equation}
                      \begin{split}
                          4x - 3y -2 & = 3x - 7y \\
                          4x - 3x    & = -7y + 3y + 2 \\
                          x    & = -4y + 2 \\
                      \end{split}
                  \end{equation}
                  Substituting $x = -4y + 2$ in equation 2:
                  \begin{equation}
                      \begin{split}
                          x + 5y - 2         & = y + 4 \\
                          (-4y + 2) + 5y - 2 & = y + 4 \\
                          -4y + 2 + 5y - 2 & = y + 4 \\
                          -4y + 5y + 2 - 2 & = y + 4 \\
                          -4y + 5y + y     & = 4 -2 + 2\\
                          2y     & = 4\\
                          y     & = \frac{4}{2}\\
                          y     & = 2\\
                      \end{split}
                  \end{equation}
                  Substituting $y = 2$ in equation 1:
                  \begin{equation}
                      \begin{split}
                          4x - 3y -2 & = 3x - 7y \\
                          4x - 3(2) - 2 & = 3x - 7(2) \\
                          4x - 6 - 2 & = 3x - 14 \\
                          4x - 3x & = -14 + 6 + 2 \\
                          x & = -8 \\
                      \end{split}
                  \end{equation}
                  Therefore $x=-8$ and $y=2$.
            \item
                  \begin{gather*}
                      \frac{1}{2}z - \frac{1}{4}w = \frac{1}{6} \ \ (eq. \ 1)\\
                      \frac{1}{2}z + \frac{1}{4}w = \frac{1}{6} \ \ (eq. \ 2)\\
                  \end{gather*}
                  Finding $z$ in equation 1:
                  \begin{equation}
                      \begin{split}
                          \frac{1}{2}z - \frac{1}{4}w & = \frac{1}{6}\\
                          \frac{1}{2}z & = \frac{1}{6} + \frac{1}{4}w\\
                          2(\frac{1}{2}z) & = 2(\frac{1}{6} + \frac{1}{4}w)\\
                          z & = \frac{2}{6} + \frac{2}{4}w\\
                          z & = \frac{1}{3} + \frac{1}{2}w\\
                          z & =  \frac{1}{2}w + \frac{1}{3} \\
                      \end{split}
                  \end{equation}
                  Substituting $z  =  \frac{1}{2}w + \frac{1}{3}$ in equation 2:
                  \begin{equation}
                      \begin{split}
                          \frac{1}{2}z + \frac{1}{4}w & = \frac{1}{6}\\
                          \frac{1}{2}( \frac{1}{2}w + \frac{1}{3}) + \frac{1}{4}w & = \frac{1}{6}\\
                          \frac{1}{4}w + \frac{1}{6} + \frac{1}{4}w & = \frac{1}{6}\\
                          \frac{1}{4}w + \frac{1}{4}w + \frac{1}{6} & = \frac{1}{6} \\
                          \frac{1}{4}w + \frac{1}{4}w & = \frac{1}{6} - \frac{1}{6} \\
                          \frac{1}{2}w & = 0 \\
                          w & = 0 \\
                      \end{split}
                  \end{equation}
                  Substituting $w = 0$ in equation 1:
                  \begin{equation}
                      \begin{split}
                          \frac{1}{2}z - \frac{1}{4}w & = \frac{1}{6} \\
                          \frac{1}{2}z - \frac{1}{4}(0) & = \frac{1}{6} \\
                          \frac{1}{2}z - 0 & = \frac{1}{6} \\
                          \frac{1}{2}z  & = \frac{1}{6} + 0\\
                          2(\frac{1}{2}z)  & = 2(\frac{1}{6}) \\
                          z &= \frac{2}{6} \\
                          z &= \frac{1}{3} \\
                      \end{split}
                  \end{equation}
                  Therefore $w=0$ and $z=\frac{1}{3}$.
        \end{enumerate}
\end{description}

\end{document}